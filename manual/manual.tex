\documentclass{article}
\usepackage{amsmath}
\usepackage{amssymb}
\usepackage{graphicx}
\usepackage{psfrag}
\usepackage{color}

\newcommand{\mtt}[1]{\mbox{\tt #1}}
\input /users/davidmcallester/03/tex/nofill.tex

\newcommand{\fopen}{{\color{red} \langle}}
\newcommand{\fclose}{{\color{red} \rangle}}

\newcommand{\ignore}[1]{}

\parskip = 2ex
\newcommand{\context}{\text{\emph{context}}}

\title{MetaC}
\author{David McAllester}

\setcounter{tocdepth}{2}

\begin{document}
\maketitle

\begin{abstract}
  MetaC provides a read-eval-print loop (a REPL) and notebook interactive development environment (a NIDE) for C programming.
  MetaC can also be described as a Lisp-inspired programming environment for C.
  The REPL and the NIDE  are capable of incrementally evaluating (compiling and running)
  C statements and procedure definitions in a persistent C process.  This is implemented by dynamically compiling dynamic load libraries (DLLs)
  (files with .so extensions) which are then linked into the running C process.

  MetaC also extends C with a universal syntax reader, computed macros, Lisp-style backquote and a pattern-matching case statement.
  The implementation is bootstrapped --- it is implemented in itself.  These C extensions are intended to support the development of frameworks
  with notebook interfaces as extensions of C rather than as scripting-C hybrids such as TensorFlow or PyTorch.
\end{abstract}

\newpage

\tableofcontents

\newpage

\section{Introduction and Overview}

\subsection{Installation}

\begin{enumerate}
\item Clone the repository into a repository directory.

\item edit the file  mc.el 
We start with hello world.

\subsection{Hello World}

Once one has cloned the MetaC repository (git@github.com:mcallester/MetaC.git) one can cd to the repository directory and type 

\begin{verbatim}
bash$ make MC
...

bash$ MC

MC>
\end{verbatim}

We can then evaluate a hello world expression.

\begin{verbatim}
MC> `{hello world}

hello world

MC>
\end{verbatim}

In the above example the backquote expression given to the REPL macro-expands to a C expression which evaluates to the expression ``hello world''.
Backquote is described in more detail below.

\subsection{C Statements, C expressions, and Universal Syntax}

We can also declare global variables and execute statements from the REPL.

\begin{verbatim}
MC> int x[10];

done

MC> for(int i = 0; i < 10; i++)x[i] = i;

done


MC> for(int i = 0; i < 10; i++)fprintf(stdout,"%d",x[i]);

0123456789done

MC>int_exp(x[5])

5

MC>
\end{verbatim}

In C syntax one distinguishes C expressions from C statements.  A C
statement is executed for effect while a C expression is evaluated for
value. The REPL can be given either a statement or an expression.  When
given a C statement, the REPL simply prints ``done'' after executing
the statement.
When given a C expression the REPL computes and prints
the value.  The REPL assumes that the value of a C expression
is a universal syntax tree (also called an expression, giving a second, and confusing,
sense of the term ``expression''). The procedure {\tt int\_exp} above
converts an integer to a universal syntax expression.  Universal syntax expressions are abstract
syntax trees but can viewed as representations of strings.

If a C statement executes a {\tt return} outside of any
procedure then that value is returned to the REPL and printed.
For example, the above session can be extended with the following.

\begin{verbatim}
MC>{int sum = 0; for(int i = 0; i < 10; i++)sum += x[i]; return int_exp(sum);}

45

MC>
\end{verbatim}

\subsection{Definitions of Types and Procedures}

One can also define new data types and procedures from the REPL.

\begin{verbatim}

MC>typedef struct myexpstruct{
   char * label;
   struct myexpstruct * car;
   struct myexpstruct * cdr;} myexpstruct, *myexp;

done

MC> myexp mycons(char*s,myexp x,myexp y){
    myexp cell=malloc(sizeof(myexpstruct));
    cell->label=s;
    cell->car=x;
    cell->cdr=y;
    return cell;}

done

MC> expptr myexp_exp(myexp x){
   if(x==NULL)return string_atom("nil");
   return
    `{${string_atom(x->label)}
      ${myexp_exp(x->car)}
      ${myexp_exp(x->cdr)}};
}

done

MC> myexp_exp(mycons("foo",mycons("bar",NULL,NULL),NULL))


foo bar nil nil nil
MC>
\end{verbatim}

Procedures can also be redefined at the REPL provided that the
signature (argument types and return type) remains the same.  In the
current implementation (June 5, 2018) changing a procedure signature
requires restarting the MetaC REPL.  This restricting ensures
meaningful C type checking in the presence of dynamic linking.

\subsection{The Global Variable Array Restriction}

To simplify dynamaic linking, all global data
variables must be arrays. One can always use single element
arrays to represent non-array variables.  To make this more convenient
we allow single element arrays to be declared with an initial value in a
manner similar to non-array data variables.  For example, we can declare and assign a variable {\tt y}
as follows.

\begin{verbatim}
MC> int y[0] = 2;

done

MC> y[0] += 1;

done

MC> int_exp(y[0])

3

MC>
\end{verbatim}

Here assignments to $y[0]$ are allowed but assignments to $y$ are not --- assignment to array variables
are not allowed in $C$. As noted above, this restriction greatly simplifies dynamic linking of data variables.

\subsection{The Single Symbol Type Restriction}

To simplify the implementation of MetaC all type expressions appearing in procedure signatures and global array declarations must be single symbols.
Arbitrary types can be given single symbol names using {\tt typedef}.

\subsection{Backquote and Universal Syntax Expressions}

We now consider backquote and universal syntax expressions in more detail.  Backquote can be used in a manner analogous to string formatting.

\begin{verbatim}
MC> expptr friend[0] = `{Bob Givan};

done

MC> int height[0] = 6;

done

MC> `{My friend ${friend[0]} is ${int_exp(height[0])} feet tall.}

My friend Bob Givan is 6 feet tall.

MC> 
\end{verbatim}

While universal syntax expressions can be viewed as representations of strings, universal syntax
expressions are actually abstract syntax trees. The tree structure
plays an important role in pattern matching. The tree structure is
more apparent in the following example.

\begin{verbatim}
MC> expptr e[0] = `{a+b};

done

MC> `{bar(${e[0]})}

bar(a+b)

MC>
\end{verbatim}

As described in the next section, the pattern {\tt foo($\$$x)} will match the expression {\tt foo(a+b)} with {\tt x} bound to the expression (tree) {\tt a+b}.

\subsection{Pattern Matching}

MetaC provides a pattern matching case construct {\tt ucase} (universal case).
The general form of the case construct is

\begin{verbatim}
ucase{e;
  {<pattern1>}:{<body1>}
   ...
  {<patternn>}:{<bodyn>}}
\end{verbatim}

Variables are marked in patterns by the prefix $\$$.

\begin{verbatim}

MC>int numeralp(expptr x)
    {if(!atomp(x))return 0;
    char*s=atom_string(x);
    for(int i=0;s[i]!='\0';i++){if(s[i]<'0'||s[i]>'9')return 0;}
    return 1;
}

done

MC> int value(expptr e)
  {ucase
    {e;
      {$x+$y}:{return value(x)+value(y);}
      {$x*$y}:{return value(x)*value(y);}
      {($x)}:{return value(x);}
      {$z}.(numeralp(z)):{return atoi(atom_string(z));}
    }
  return 0;
  }

done

MC>int_exp(value(`{5+2*10}))

25

MC>int_exp(value(`{foo}))


 match error: the value 


foo
does not match any of


{$z}.(numeralp(z)){($x)}{$x*$y}{$x+$y}

MC>
\end{verbatim}

In many situations the choice of the particular tree structure imposed by the MetaC reader is not important as long as the same conventions are used in both the patterns and the expressions
they are matching.  For example, the expression {\tt a+b+c} will match the pattern
{\tt $\$$x+$\$$y+$\$$z} independent of whether + is left associative or right associative.  But the tree structure does matter in other cases.  The pattern {\tt $\$$x*$\$$y}
will not match the expression {\tt a+b*c} because the reader brackets {\tt a+b*c} as  $\fopen \mathtt{a~+} \fopen \mathtt{b * c} \fclose\fclose$. The pattern {\tt $\$$x*$\$$y} does match {\tt (a+b)*c}.

When a {\tt ucase} is executed the patterns are tries sequentially and stops after the first match.  If no pattern matches an error is generated.  One can always add a final pattern of $\{\$x\}$ to provide a default
catch-all case if that is desired.

As of June 2018 repeated variables, such as in the pattern {\tt \{$\$$x + $\$$x\}}, are not supported (and not checked for).
If a variable is repeated it will be matched at its last occurrences under depth-first left to right traversal.

\subsection{Gensym and Macro Definitions}

MetaC supports computed macros which can exploit pattern matching, backquote pattern instantiation, and other (bootstrapped) high level language features.
As a very simple example we can define a {\tt dolist} macro at the REPL.

\begin{verbatim}

MC>umacro{mydolist($x,$L){$body}}{
   expptr rest=gensym("rest");
   return
    `{for(explist$rest=$L; cellp($rest);$rest=cdr($rest);){
      expptr$x=car($rest);
      $body}};
}

done

MC>macroexpand(`{mydolist(item,list){f(item);}})

for
  (explist _genrest33=list;
   cellp(_genrest33);
  _genrest33=cdr(_genrest33);
  ){expptr item=car(_genrest33);f(item);}

MC>
\end{verbatim}

The general form of a macro definition is
\begin{verbatim}
umacro{<pattern>}{<body>}
\end{verbatim}
where instances of {\tt <pattern>} in MetaC code are to be replaced by the value returned by {\tt <body>} under the variable bindings determined by the match.
The procedure {\tt macroexpand} takes a universal syntax expression
and returns the result of repeatedly expanding outermost macros until no further macro expansion is possible.  Macro expansion can generate effects as well as return an expansion.  The REPL performs
macro expansion on the given expression and also performs the effects of that expansion.  {\tt umacro} is itself a MetaC macro and we can feed the above macro definition to the REPL.

The macro expansion of the above macro definition defines a procedure to compute the macro expansion and installs that procedure on a macro property of the symbol {\tt mydolist}.
In a typical use a macro patterns is a symbol followed by a sequence of parenthesis expressions., in which case the expansion procedure is attached to the given symbol,
or generalized application expression or a binary connection expressions.  Universal syntax expression types are discussed in the next section.
The procedure for macro expansion is attached to either the head symbol of the application
or the binary operator of the binary connection expression.

\subsection{A Notebook IDE}

The MetaC interactive development environment (IDE) unifies the concept of a source file with the concept of a notebook.
During active code development a .mc file behaves as a notebook. Once the code is mature, a .mc file can be macro-expanded to a .c
file which can then be compiled to an object file and linked into an executable the the traditional way.  Extending the REPL executable by linking additional object files
is discussed in section~\ref{sec:extensions}.

An emacs-based interactive development environment (IDE)
is included in the MetaC repository in the file mc.el.  This file should be loaded form the users .emacs file so that the the proper hooks run when opening .mc files.
Files with extention .mc open in c-mode extened with MetaC commands.

The two fundamental IDE commands are C-M-s, which starts or restarts a kernel, and C-cC-c which executes a the ``cell'' at the current cursor position.  A cell is defined
by the MetaC segementation convention under which a cell starts at a non-empty line whose first character is other than a white space character, a close paren character or
the slash character.  The cell ends at the beginning of the next non-empty line whose first character is other then white space or close paren character.  Note that lines starting
with a slash (typically a comment) terminates a cell but does not start a new cell.
In the current implementation (July 2018) it is important that comments lines either start with a slash or start with some non-empty indentation.
Clearly there is room for improvement in the parsing of .mc files in the IDE, but the current version is usable as long as one cognizant of the conventions.

Starting the kernel with the command C-M-s creates a MetaC REPL process associated with the buffer *MetaC*.  The REPL can be run as usual from that buffer
but the REPL buffer typically remains hidden when treating a .mc file as a notebook.  When a cell is executed with the command C-cC-c the text of the cell is (invisibly) passed to the REPL
and the text returned by the REPL is placed inside a comment below the cell.  If a comment already occurs directly below the cell this comment is replaced by a new comment containing the
response of the REPL.  As in a Jupyter notebook, the values returned from the REPL are numbered so that one can see when the cell was executed relative to other cell execution in the notebook.
If C-M-s is run when a kernel is already running the kernel process is killed and a new one started (as in a Jupyter notebook).  When the kernel is restarted
the execution numbering is reset to start with 1.  Changing a procedure signature (argument and return types) requires restarting the kernel.

If an error occurs doing the execution of a cell the comment following the cell states that an error occured and suggests running the command C-M-m to switch to the *MetaC* buffer
to see the error message and possibly interact with the REPL as a debugger.  Debugging features are described in section~\ref{sec:debugger}.

In addition to starting (or restarting) the kernel and to executing individual cells, we have the following commands.

\begin{itemize}
\item[C-cr] This executes all cells in the current region sequentially with a different execution number for each cell execution.

\item[C-M-a] This moves to the begging of the current cell.

\item[C-M-p] This moves to the beginning of the preceding cell.

\item[C-M-n] This moves to the beginning of the next cell.

\item[C-M-m] This goes to the *MetaC* buffer so that one can see error messages and interact with the REPL as a debugger.  See section~\ref{sec:debugger} for a discussion of debugging features.

\item[C-M-c] This goes back from the *MetaC* buffer (which is in shell-mode) to the last .mc file from which C-M-m was run.
\end{itemize}

\subsection{Extending the REPL executable}
\label{sec:exntensions}

Comming soon.

\subsection{Debugging Features}
\label{sec:debugging}

Coming soon.

\section{Universal Syntax}

It is not obvious how to implement light weight expression quotation and expression pattern matching for C expressions.  The syntax of C is complex.
We bypass this complexity by introducing a syntax with the following three ``universal'' properties.

\begin{enumerate}
\item Universal syntax trees are semantics-free.  They are simply trees viewed as representations of character strings.

\item The universal syntax reader can read any parenthesis-balanced character string.
  Here parenthesis-balanced means that every open parenthesis, brace or bracket has a matching closing character and that string quotations
  are properly closed.
  
\item The printer inverts the reader. If we read a string $s$ into a universal syntax expression
  $e$ and then print $e$ back into a string $s'$ we have that $s$ and $s'$ are equivalent in a strong sense.  Here we want
  $s$ and $s'$ to be ``whitespace equivalent'' so as to be treated equivalently by the C lexer.
\end{enumerate}

The emphasis here is on the representation of strings.  The reader does not always invert the printer ---
the expression $\fopen \fopen \mathtt{one~two} \fclose \;\mathtt{three}\fclose$ prints as {\tt one two three} which reads as
$\fopen \mathtt{one}\; \fopen \mathtt{two~three} \fclose \fclose$.  But the represented string is preserved.  This is fundamentally different from
most programming languages supporting symbolic computation (such as Lisp) where it is assumed that the tree structure, rather than the represented string, is fundamental
and hence that the reader should invert the printer.

The above universality properties allow one to assign semantics to universal syntax expressions based on the strings that they represent.  We can assign C
semantics to an expression by printing the expression and passing the resulting string to a C compiler.  This string-based semantics is not always
compositional with respect to the universal syntax bracketing.  However, the tree structure imposed by the reader is designed to approximate the
compositional structure of C syntax.  In most cases pattern matching on universal syntax expressions recovers substructure that is semantically meaningful under C
semantics. Parentheses and semicolons are particularly helpful in aligning universal syntax trees with C syntax.

\subsection{Universal Expressions: cons-car-cdr}

A universal syntax expression is either an atom (a wrapper around a string),
a pair $\fopen e_1 e_2 \fclose$ where $e_1$ and $e_2$ are expressions, or a ``parenthesis expression'' the form $(e)$,
$\{ e\}$, or $[e]$ where $e$ is an expression.   All three of these datatypes have the same C type {\tt expptr} (Expression Pointer).
Each has a constructor procedure, a predicate, and accessor functions.  The constructor, predicate and accessors for atoms, pairs and parenthesis expressions
are the following.
The atom data type has the constructor, test and accessor function for atoms are the following.

\begin{verbatim}
expptr string_atom(char *);
int atomp(expptr);
char * atom_string(expptr); //the argument must be an atom.

expptr cons(expptr,expptr);
int cellp(expptr);
expptr car(expptr); //the argument must be a cell. returns the first component.
expptr cdr(expptr); //the argument must be a cell. returns the second component.

expptr intern_paren(char,expptr); // the char must be one of '(', '{' or '['
int parenp(expptr);
expptr paren_inside(expptr);
\end{verbatim}

\subsection{Universal Expressions: Phantom Brackets}

The MetaC reader maps a character string to a universal syntax expression.  A specification of the reader is given in section~\ref{sec:reader}.
The expression produced by the reader
can be represented by ``phantom brackets'' around the given string
where there is a pair of brackets for each cons cell.
For example the expression \mtt{\{one two three\}} reads as $\{\fopen \mathtt{one}\;\fopen \mathtt{two~three}\fclose \fclose\}$.
Examples of strings and the phantom bracking imposed by the reading those strings is given in figure~{ref{fig:reader}}.
The printer simply removes the phantom brackets and prints the string that the expression represents.

To reduce the clutter of brackets we will adopt two conventions for supressing brackets when exhibiting phantom brackets.
The first convention is the Lisp right-branching convention of not showing all the cell brackets for right-associative (right-branching) sequences.
For example $\fopen \mathtt{zero}\;\fopen \mathtt{one}\;\fopen \mathtt{two~three}\fclose \fclose \fclose$.
will be written as $\fopen\mathtt{zero~one~two~three}\fclose$.  Note that these ``lists'' are atom-terminated rather than the Lisp convention of
nil termination.

The second convention is that we will sometimes write $\fopen \fopen a_1 \;o\fclose\;a_2\fclose$ where $o$ is a connective atom
as the left-branching structure $\fopen a_1\;o\;a_2 \fclose$.  For example, the expression \mtt{\{a + b\}} reads as
$\{\fopen \fopen \mathtt{a~+} \fclose \;\mathtt{b} \fclose \}$ which is then abbreviated as
$\{\fopen \mathtt{a~+~b} \fclose \}$.  Left-branching for binary connectives gives a
C-consistent treatment of semicolon as a binary connective while also supporting the interpretation of semicolon as a statement terminator.
This is discussed in more detail below.

It will generally be clear from context whether we using the Lisp right-branching list convention
or the left-branching connective convention.
A set of examples of strings and the phantom brackets generated by the reader is given in figure~\ref{fig:reader}.


To emphasize the significance of left-branching binary connectives we note that
the MetaC reader bracketing of
$$\{e_1\;;\;e_2\;;\;e_3\;;\;e_4\}$$
can be written using either the left-braching binary connective convention as
$$\{\fopen e_1\;;\;\fopen e_2\;;\;\fopen e_3\;;\;e_4\fclose\fclose\fclose\}$$
or the right-braching sequence convention as
$$\{\fopen \fopen e_1\;;\fclose \;\fopen e_2\;;\fclose \;\fopen e_3\;;\fclose\;e_4\fclose\},$$
both of which abbreviate the same full bracketing
$$\{\fopen \fopen e_1\;;\fclose \;\fopen \fopen e_2\;;\fclose \;\fopen \fopen e_3\;;\fclose\;e_4\fclose\fclose\fclose\}.$$

\begin{figure}
\begin{eqnarray*}
  \mathtt{Hello~World} & \Rightarrow & \fopen \mathtt{Hello}\;\; \mathtt{World}\fclose \\
  \mathtt{one~two~three} & \Rightarrow & \fopen \mathtt{one}\;\; \fopen \mathtt{two} \; \;\mathtt{three}\fclose\fclose \\
   & = & \fopen \mathtt{one~two~three} \fclose \\
  \mathtt{x~+~y} & \Rightarrow & \fopen\fopen \mathtt{x~+}\fclose\;\mathtt{y}\fclose \\
  & = & \fopen \mathtt{x~+~y} \fclose \\
  \mathtt{x+y*z} & \Rightarrow & \fopen \mathtt{x~+}\; \fopen \mathtt{y~*~z} \fclose\fclose \\
  \mathtt{(x+y)*z} & \Rightarrow & \fopen (\fopen \mathtt{x~+~y} \fclose )\;\mathtt{*~z}\fclose \\
  \mathtt{foo}(\mathtt{int~x}) & \Rightarrow & \fopen \mathtt{foo}\; (\fopen \mathtt{int}\;  \; \mathtt{x}\fclose)\fclose \\
  \mathtt{foo(int~x,~float~y)} & \Rightarrow & \fopen \mathtt{foo} \; (\fopen\fopen \mathtt{int}\; \mathtt{x} \fclose\;
  \mathtt{,}\; \fopen\mathtt{float} \; \mathtt{y} \fclose\fclose )\;\fclose
\end{eqnarray*}

\caption{{\bf Examples of Reader Bracketings.} Bracketings are shown for the expression that results form reading the given strings.
  A complete bracketing shows a pair of brackets for every expression pair (cons cell).
  The second and third example show two somewhat informal conventions for dropping some of the brackets --- general sequences are
  assumed to be right-associative and binary connective applications are assumed to be left-associative.  These conventions are used in other examples.
  Expressions are printed without the brackets --- the brackets are ``phantoms'' that show the tree structure.   The bracketing (parsing) done by the reader
  is specified formally in section~\ref{sec:reader}. Since the bracketing does not affect the represented string, the precise bracketing is often unimportant.}
\label{fig:reader}
\end{figure}


\subsection{The Reader Specification}
\label{sec:reader}

The MetaC reader can be described in three phases --- preprocessing, lexicalization, and parsing.

The MetaC preprocessor removes C-style comments and divides files into segments where each segment is to be read as a separate expression.
A new segment starts at the beginning of any line whose first character is not a space, tab or close character (right parenthesis, brace or bracket).
A file must be parenthesis-balanced within each segment.
For the REPL a segment ends at the first return character not occurring inside parentheses.

Lexicalization segments a pre-processed character sting into to a sequence of atoms.
The MetaC lexer preserves all non-white characters.  For the MetaC lexer each atom is one of the following.

\begin{itemize}
\item Symbols,  A symbol is a character string consisting of only alpha-numeric characters --- upper and lower case letters of the alphabet, plus the decimal numerals, plus underbar.
  For example {\tt foo\_bar1}.
\item Connectives.  A connective is a character string consisting of only ``connective characters'' --- the connective characters consist of all non-alphanumeric, non-white, non-parenthesis, characters
  other than the three special characters {\tt \$}, {\tt `} and $\backslash$.
\item Quoted strings.  A character string starting and ending with the same string quotation character.  For example {\tt "foo bar"}, {\tt "!:\&?;"} or {\tt 'a'}.
\item Single character atoms for parenthesis characters and the three special characters $\mtt{`}$, $\mtt{\$}$, and $\backslash$.
\end{itemize}

Two strings will be called whitespace-equivalent if they lexicalize to the same sequence of atoms.
Two strings that lexicalize equivalently under the MetaC lexer will also lexicalize equivalently under the C lexer.

\newcommand{\sym}{\mathrm{SYM}}
\newcommand{\conn}{\mathrm{CONN}}
\newcommand{\quot}{\mathrm{QUOTE}}
\newcommand{\misc}{\mathrm{MISC}}
\newcommand{\app}{\mathrm{APP}}
\newcommand{\var}{\mathrm{VAR}}



The reader is specified by a the grammar shown in figure~\ref{fig:grammar}.
To simplify the specification of the reader we enclose the given string in parentheses so that, without loss of generality, the input is assumed to be of the form
$\{s\}$ where $s$ is the lexical sequence to be read.  The grammar has a nonterminal $P$ for parenthesis expression which we take as the top level nonterminal.
The nonterminals $\sym$ and $\quot$ range over alpha-numeric string atoms and quoted string atoms respectively.
For each positive integer $p$ we have a nonterminal $\conn_p$ ranging over connective atoms of precedence $p$.
We have a nonterminal $E_p$ for expressions formed with connectives
of precedence $p$.  We also has a nonterminal $E_\infty$ for expressions that are formed at higher precedence than any connective expression.
The precedence levels are divided into a set ${\cal L}$ of left-associative precedence levels and a set ${\cal R}$ of right associative precedence levels.
There is a special precedence level 3 for combining expressions with the ``null connective''. The null connective is left-associative.
The null connective is higher precedence that semicolon and comma but lower precedence than all other connectives.
The nonterminal $E_\infty$ includes an epsilon production allowing null arguments.

\begin{figure}
  
\begin{eqnarray*}
  P & ::= & (E_p) \;|\; \{E_p\} \;|\; [E_p] \\
  E_p & ::= & \fopen \fopen E_\ell\;\conn_p\fclose \;E_r\fclose\;\;\;\;p \in {\cal R}, \;\ell > p, \; r \geq p \\
  E_p & ::= & \fopen \fopen E_\ell\;\conn_p\fclose \;E_r\fclose\;\;\;\;p \in {\cal L}, \; \ell \geq p,\; r > p \\
  E_3 & ::= & \fopen E_\ell \;E_r\fclose \;\;\;\ell > 3,\;r \geq 3\\
  E_\infty & ::= & \quot \;|\; \fopen S \; \;P^* \fclose \;|\; \fopen `\;P\fclose \;|\; P \;|\; \epsilon \;|\; \mathrm{JUNK}\\
  S & ::= & \;\sym \;|\; V \\
  V & ::= & \fopen \$\;\sym\fclose \;|\; \fopen \$\;P\fclose \;|\; \fopen \backslash \; V \fclose\\
  P^* & ::= &  \epsilon  \;|\; \fopen P\;P^* \fclose \\
  \mathrm{JUNK} & ::= & ` \;|\; \mathrm{SJUNK} \\
  \mathrm{SJUNK} & :: = & \$ \;|\;  \backslash \;|\; \fopen \backslash\;\mathrm{JUNK} \fclose\\
\end{eqnarray*}

\caption{{\bf The grammar defining the MetaC reader.} The input string is assumed to be of the form $(s)$ so that endpoints are explicitly labeled.
The nonterminal $\sym$ generates non-empty alpha-numeric strings; $\conn_p$ generates non-empty connective
strings of precedence $p$; and $\quot$ generates string quotations.  ${\cal R}$ is a set of right-associative precedence levels and ${\cal L}$ is a set of left-associative levels.
Only the highest precedence level is left-associative. MetaC classifies all printable ASCII characters as being either alpha-numeric, connective, string quotations, parenthesis characters or one of the three special characters
{\tt `}, $\$$ or $\backslash$.
The reader will accept any parenthesis-balanced string. $\epsilon$ denotes the empty string. Generated cells of the form $\fopen w\;\epsilon \fclose$ or $\fopen \epsilon \;w\fclose$ are replaced by $w$.
Although the grammar is ambiguous, the reader is deterministic as specified by the constraints that
$E_\infty$ expressions must be maximal and that the use of the $\epsilon$ production for $E_\infty$ must be minimized.  See the text for details.
  }
\label{fig:grammar}
\end{figure}

The nonterminal $S$ ranges over symbols or variables where $V$ ranges over variables.  Variables are handles specially in pattern matching ({\tt ucase}) and pattern instantiation (backquote).
Expression of the form $\fopen S\;P^*\fclose$ include {\tt foo}, {\tt foo(x)}, {\tt foo(int x)\{return x;\}},
{\tt \$x}, {\tt \$f(x)}, and {\tt \$\{f(x)\}}.


The above grammar is ambiguous.  For example the string {\tt \{foo (a) (b)\}} can be parsed as either the left-branching structure $\{\fopen\fopen\sym\;P^*\fclose\;P\fclose\}$
or the  right-branching structure $\{\fopen\sym\;P^*\fclose\}$.
The reader is of course deterministic --- {\tt \{foo (a) (b)\}} is read as right-branching.  While a deterministic grammar for the reader can be given, it is simpler
to specify a deterministic parsing process for the above ambiguous grammar.
The implementation runs a deterministic left-to-right shift-reduce process.
However, the parser is easier to specify as operating in global stages.  In the first stage one identifies all maximal $E_\infty$ substrings.
The maximal $E_\infty$ requirement specifies that {\tt \{foo (a) (b)\}} is parsed as a single (right-branching)
$E_\infty$ expression. However the string {\tt \{$\$$\;foo\;(a)\;(b)\}} is read as the longer $E_\infty$ expression $\{\fopen \fopen \mathtt{!~foo}\fclose\ \fopen \mathtt{(a)~(b)} \fclose \fclose$.
The grammar does have the property that any given string has a unique segmentation into maximal $E_\infty$ expressions each of which has a unique structure.

After identifying maximal $E_\infty$ expressions we are left with a sequence of arguments (the $E_\infty$ expressions) and connectives.
However, it is possible that this sequence contains multiple consecutive connectives or multiple consecutive arguments.  We then place an argument between any two consecutive
connectives using the $\epsilon$ production for $E_\infty$ and also add an $\epsilon$ term at the beginning of the sequence if the sequence starts with a connective and an
epsilon argument at the end if the sequence ends in a connective.  We now have a sequence of arguments and connectives starting and ending with arguments and where there
are no two consecutive connectives.  Next set $k$ to be the largest precedence of any connective and form all the $E_k$ substrings. We then repeatedly decrement $k$ and identify all the $E_k$ substrings
until we have identifies all $E_p$ substrings for all $p$.  At this point the entire string must be parsed as $E_p$ where $p$ is the smallest precedence of the connectives
where we think of adjacent arguments as having an implicit precedence 3 connetive between them.

\noindent This process can parse any parenthesis-balanced string.

The above specification of the reader can result in cells of the form $\fopen e\;\epsilon \fclose$ or $\fopen \epsilon \;e\fclose$.   The implementation avoids producing such cells
and the reader can be viewed as replacing $\fopen e\;\epsilon \fclose$ or $\fopen \epsilon \;e\fclose$ by $e$ so that
all cells are of the form $\fopen e_1\;e_2\fclose$ where $e_1$ and $e_2$ are expressions representing non-empty strings.  An empty parenthesis string {\tt ()}
is read as a parenthesis expression containing an atom for the empty string.

\noindent Grouped by precedence, the binary connective characters are

\medskip
\centerline{\{\mtt{;}\} \{\mtt{,}\}  \{\mtt{|}\} \{$\epsilon$\}
  \{{\tt \&}, {\tt ?}, {\tt !}\} \{{\tt =} {\tt $\sim$}, {\tt $<$}, {\tt $>$}\} \{{\tt +}, {\tt -}\}
  \{{\tt *}, {\tt /}\} \{{\tt \%}, $\hat{~}$, {\tt .}\} \{\mtt{:},  {\tt @}, {\tt \#}\}.}

\noindent The precedence of a binary connective is determined by its first character.  The
characters have low to high (outer to inner) precedence in the order given with symbols in the same
group having the same precedence.  Here $\epsilon$ represents the null connective at precedence
level 3.  The precedence can be summarized in outer to inner order as semicolon, comma, bar, the
null connective, Boolean connectives, binary predicates, three levels of binary functions, and
innermost connectives which can be viewed as providing a way of building structured atoms.  This
universal syntax precedence conventions has
various divergences with C syntax.  However, when writing a macro package in MetaC one can simply
adopt the MetaC precedence conventions for the source code and use parentheses where necessary in
the generated C code.

\subsection{Backquote and Pattern Matching Revisited}

The semantics of backquote is defined in terms of cons-car-cdr view of expressions independent of the MetaC reader.
In the typical case, the C value of a backquote expression
$\fopen ` \;\{e\}\fclose$ is the expression (tree) $e$ with subexpressions (subtrees) of the form $\fopen \$\;x \fclose$, where $x$ is a symbol,
replaced by the C value of $x$
and subexpressions (subtrees) of the form $\fopen \$\;\{w\}\fclose$ replaced by the expression which is the C value of the string represented by the expression $w$.

Unfortunately this evaluation rule for backquote expressions is incomplete.  One of the most
confusing situations is where the expansion of a macro contains a backquote.  Writing such a macro typically involves nested backquotes.  While nested backquotes
are confusing, and should be avoided when possible, MetaC supports nested backquotes.  We consider a series of backquote expressions each of which evaluates to the previous one.

First we have
\begin{equation}
\label{bq1}
`\{a+b+\$\{z\}\}
\end{equation}

If the value of variable {\tt z} is the expression {\tt c} then the value of expression (\ref{bq1}) is the expression {\tt a+b+c}.
The symbol {\tt \$} can be included in the value of a backquote expression by quoting it.  This gives our second expression.
\begin{equation}
\label{bq2}
`\{`\{a + \$\{y\} + \backslash\$\{z\}\}\}
\end{equation}

If the value of variable {\tt y} is the expression $b$ then the value expression of (\ref{bq2}) is expression (\ref{bq1}).
We can even have multiple layers of quotation as in the following.
\begin{equation}
\label{bq3}
`\{`\{`\{\$\{x\} + \backslash\$\{y\} + \backslash\backslash\$\{z\}\}\}\}
\end{equation}

If the value of variable {\tt x} is the expression {\tt a} then the value of expression (\ref{bq3}) is expression (\ref{bq2}).

As with backquote, pattern matching is defined in terms of the cons-car-cdr view of expressions independent of the MetaC reader.  We define a substitution to be a mapping from
symbol atoms to expressions. For a substitution $\sigma$ and a pattern $\{e\}$ we define the expressions $\sigma(e)$
to be the result of replacing each subexpression (subtree) of $e$ of the form $\fopen \$\;x \fclose$ where $x$ is a symbol atom with $\sigma(x)$.  A pattern expression (tree) $\{e\}$ matches an expression (tree) $w$ with substitution $\sigma$
if $\sigma(e) = w$. An exception to this is the case where a variable occurs multiple times in the pattern.  As of June 2018 multiple occurrences of a variable in a pattern is not supported.






\end{document}
